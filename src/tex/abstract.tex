\textbf{\emph{Introduction}}



Software development is as much of an art as it is science. It is not only the application of one's logic but also the expression of their imagination and craftsmanship. Though it might not seem like a big deal for a new developer to create working applications, gaining mastery of this craft requires passion, dedication, pragmatism and continuous learning. This handbook has been designed to provide you with a concise introduction to the best practices that one should follow in their road to mastering the art of software development. Though it may not be possible to precisely quantify the value of these best practices, they are observed to be commonly used by the best of the developers and successful organizations. 

While it is not feasible to cover every topic in depth, the handbook intends to provide you with a starting point. You should dive deeper on your own to gain an in-depth knowledge of the best practices and to implement them in real time scenarios. You might find dozens of books for many of the topics covered in the handbook. From a beginner's perspective, it might be overwhelming to digest everything covered in the handbook at once. Hence, multiple reads at certain intervals is encouraged.  

Also, this excerpt from Josh Bloch’s seminal book Effective Java is equally applicable to this handbook: 

“While the rules in this handbook do not apply 100 percent of the time, they do characterize best programming practices in the great majority of cases. You should not slavishly follow these rules, but violate them only occasionally and with good reason. Learning the art of programming, like most other disciplines, consists of first learning the rules and then learning when to break them.”

Finally, you're bound to make mistakes or errors in judgement no matter how much you try to adhere to the best practices. Learn from your mistakes and take the responsibility to rectify them. Be open to admit ignorance or error. Or to even admit that you need help. 

\textbf{\emph{Who is this handbook for?}}

This handbook is primarily written for software developers who are also referred to as software engineers or programmers. These terms mean different things but this handbook uses these terms interchangeably.  This book can also be used by people other than developers. For example a Quality Assurance Engineer can use this to know about core weaknesses of a project which leads to bugs and other problems. Similarly, a Project Manager can read this to know what developers should do to improve the health of the project  



